

\section{Screening Method}
\subsection{Problem Formulation}

The elastic net problem is the optimization of the following problem:
\begin{equation}
    \label{eq:enet}
    \underset{\beta\in \mathbb{R}^p}{\mathrm{min}}\frac{1}{2}||\boldsymbol y-X\boldsymbol\beta||_2^2+n\alpha\lambda||\boldsymbol\beta||_1+\frac{n(1-\alpha)}{2}\lambda||\boldsymbol\beta||_2^2,
\end{equation}

where $\boldsymbol\beta\in\mathbb{R}^p$ is the coefficient vectors for the $p$ features, $X=[\boldsymbol x_1,\boldsymbol x_2,...,\boldsymbol x_p]\in\mathbb{R}^{n\times p}$ is the feature matrix and $\lambda>0$ is the penalty parameter that controls the size of elastic net penalty. $\alpha\in(0,1)$ is the proportion of Lasso penalty in the elastic net penalty and $1-\alpha$ is the proportion of ridge penalty. A dual form of problem above can be given by (please see appendix for details):

\begin{gather}
        \label{eq:dualtheta}
        \underset{\boldsymbol\theta\in \mathbb{R}^{ n},\boldsymbol\gamma\in\mathbb{R}^p}{\mathrm{max}}g_\lambda(\boldsymbol\theta,\boldsymbol\gamma)\equiv\frac{1}{2}||\boldsymbol y||_2^2-\frac{\lambda^2}{2}\left\Vert\boldsymbol\theta-\frac{\boldsymbol y}{\lambda}\right\Vert_2^2-\frac{\lambda^2}{2}||\boldsymbol\gamma||_2^2\\
        \begin{aligned}s.t.\quad (\boldsymbol\theta,\boldsymbol\gamma)\in \mathcal{F}_\lambda\equiv\{(\boldsymbol\theta,\boldsymbol\gamma):\quad
            &||X^T\boldsymbol\theta-\sqrt{n(1-\alpha)\lambda}\boldsymbol\gamma||_\infty\leq n\alpha\}\nonumber,
        \end{aligned}
\end{gather}

where $\boldsymbol\theta\in \mathbb{R}^{n}$ and $\boldsymbol\gamma\in\mathbb{R}^p$ are the dual variables. The dual problem becomes the minimization of the convex function $g(\boldsymbol\theta,\boldsymbol
\gamma)$ within a convex feasible set $\mathcal{F}_\lambda$. Let $\boldsymbol\beta_\lambda$ denote the solution to the primal problem at penalty parameter value $\lambda$ and $\boldsymbol\theta_{\lambda},\boldsymbol\gamma_\lambda$ denote the corresponding dual solution. The primal solution and dual solution can be connected by

\begin{equation}
    \label{eq:dualprimal}
    \boldsymbol\theta_\lambda=\frac{\boldsymbol y-X\boldsymbol\beta_\lambda}{\lambda},\quad \boldsymbol\gamma_\lambda=\sqrt{\frac{n(1-\alpha)}{\lambda}}\boldsymbol\beta_\lambda,
\end{equation}

and the KKT conditions for the primal problem~\eqref{eq:enet} can be expressed as:

\begin{gather}
    \label{eq:kkt}
    \begin{aligned}&\boldsymbol\beta_{\lambda,j}=0\implies|\boldsymbol x_j^T\boldsymbol\theta_\lambda|\leq n\alpha\\
    & \boldsymbol\beta_{\lambda,j}\neq0\implies  \boldsymbol x_j^T\boldsymbol\theta_\lambda-n(1-\alpha)\boldsymbol\beta_{\lambda,j}=n\alpha\textit{sign}(\boldsymbol\beta_{\lambda,j}).
    \end{aligned}
\end{gather}

for any $j$. Combining \eqref{eq:dualprimal} and \eqref{eq:kkt} we have a trivial closed form solution for the problem at some $\lambda$ values:

\begin{gather}
    \label{eq:lammax}
    \begin{aligned}
        \boldsymbol\beta_\lambda=0\iff \lambda \geq \lambda_{\max}\equiv \max_j \frac{|\boldsymbol x_j^T\boldsymbol y|}{n\alpha}
    \end{aligned}
\end{gather}

Thus, when solving the problems on a grid of $L+1$ decreasing $\lambda$ values: $\lambda_0>\lambda_1>...>\lambda_L>0$, it makes more sense to choose $\lambda_0= \lambda_{\max}$ to take advantage of the known solution. If an algorithm solve the problems sequentially in decreasing order of $\lambda$, then solution at $\lambda_{l-1}$ will be known before solving the problem at $\lambda_l$. In the rest of the paper, we will derive screening method for such a path-wise structure. Also, without loss of generality, we will derive the screening method for the problem at $\lambda_1$ assuming the solution at $\lambda_0$ is known, the same method can be applied to any pair of $\lambda_{l}$ and $\lambda_{l-1}$.

The KKT conditions \eqref{eq:kkt} also say that for any $\lambda$ if 

\begin{equation}
    \label{eq:disc_cond}
    |\boldsymbol x_j^T\boldsymbol\theta_{\lambda}|<n\alpha,
\end{equation}

we can safely conclude $\boldsymbol\beta_{\lambda,j}=0$ and the corresponding $\boldsymbol x_j$ can be discarded for the optimization at $\lambda$. Although the left hand side of \eqref{eq:disc_cond} is unknown at $\lambda_1$ until the solution is obtained at $\lambda_1$, we can use the solution at $\lambda_{0}$ to derive some bound for the left hand side: $T_j(\lambda_{1},\lambda_{0};\boldsymbol\theta_{\lambda_0},\boldsymbol\gamma_{\lambda_0})\geq |\boldsymbol x_j^T\boldsymbol\theta_{\lambda_1}|$ and then if $T_j(\lambda_{1},\lambda_{0};\boldsymbol\theta_{\lambda_0},\boldsymbol\gamma_{\lambda_0})<n\alpha$ we can also safely conclude $\boldsymbol\beta_{\lambda_1,j}=0$. The goal is to find a smaller $T_j(\lambda_{1},\lambda_{0};\boldsymbol\theta_{\lambda_0},\boldsymbol\gamma_{\lambda_0})$ to make more discards. To construct such a bound, we consider the intermediate dual variable:

\begin{gather}
        \label{eq:dualmi}
        (\boldsymbol\theta_{\lambda_1|\lambda_0},\boldsymbol\gamma_{\lambda_1|\lambda_0})\equiv\underset{\boldsymbol\theta\in \mathbb{R}^{ n},\boldsymbol\gamma\in\mathbb{R}^p}{\mathrm{arg\,max}}g_{\lambda_1}(\boldsymbol\theta,\boldsymbol\gamma)\\
        \begin{aligned}s.t.\quad (\boldsymbol\theta,\boldsymbol\gamma)\in \mathcal{F}_{\lambda_0}\nonumber.
        \end{aligned}
\end{gather}

Compared to the origin dual problem \eqref{eq:dualtheta} at $\lambda_0$, the intermediate problem \eqref{eq:dualmi} optimizes a slightly different dual function on the same feasible set, while the origin dual problem at $\lambda_1$ optimize the same dual function but on a slightly different feasible set, compared to the intermediate problem. Then we can use the solution $(\boldsymbol\theta_{\lambda_0},\boldsymbol\gamma_{\lambda_0})$ to find a bound such that $(\boldsymbol\theta_{\lambda_1|\lambda_0},\boldsymbol\gamma_{\lambda_1|\lambda_0})\in \mathcal{A}^1(\lambda_1,\lambda_0|\boldsymbol\theta_{\lambda_0},\boldsymbol\gamma_{\lambda_0})$ and after that based on $(\boldsymbol\theta_{\lambda_1|\lambda_0},\boldsymbol\gamma_{\lambda_1|\lambda_0})$ find a second bound such that $(\boldsymbol\theta_{\lambda_1},\boldsymbol\gamma_{\lambda_1})\in \mathcal{A}^2(\lambda_1,\lambda_0|\boldsymbol\theta_{\lambda_1|\lambda_0},\boldsymbol\gamma_{\lambda_1|\lambda_0})$. Last if we find a bound satisfying:

\begin{equation}
    \label{eq:boundbound}
    T_j(\lambda_{1},\lambda_{0};\boldsymbol\theta_{\lambda_0},\boldsymbol\gamma_{\lambda_0})\geq \underset{(\boldsymbol\theta',\boldsymbol\gamma')\in\mathcal{A}^1(\lambda_1,\lambda_0|\boldsymbol\theta_{\lambda_0},\boldsymbol\gamma_{\lambda_0})}{\mathrm{max}}\,\underset{(\boldsymbol\theta,\boldsymbol\gamma)\in\mathcal{A}^2(\lambda_1,\lambda_0|\boldsymbol\theta',\boldsymbol\gamma')}{\mathrm{max}}|\boldsymbol x_j^T\boldsymbol\theta|,
\end{equation}

then it automatically satisfies $T_j(\lambda_{1},\lambda_{0};\boldsymbol\theta_{\lambda_0},\boldsymbol\gamma_{\lambda_0})\geq |\boldsymbol x_j^T\boldsymbol\theta_{\lambda_1}|$ and can be used for screening.

\subsection{Origin and Intermediate Dual Variables Bounds}

In this section, we will derive the two bounds $\mathcal{A}^1$ and $\mathcal{A}^2$.

\subsubsection{Method I}

Looking at the form of the dual function \eqref{eq:dualtheta}, $(\boldsymbol\theta_{\lambda_0},\boldsymbol\gamma_{\lambda_0})$ is the projection of $(\frac{\boldsymbol y}{\lambda_0},0)$ onto $\mathcal{F}_{\lambda_0}$, while in the intermediate problem \eqref{eq:dualmi}, $(\boldsymbol\theta_{\lambda_1|\lambda_0},\boldsymbol\gamma_{\lambda_1|\lambda_0})$ is the projection of a different point $(\frac{\boldsymbol y}{\lambda},0)$ onto the same set $\mathcal{F}_{\lambda_0}$. Using properties of projection onto a convex set as in the enhanced dual polytope projection (EDPP) \citep{wang2013lasso}, a bound for $(\boldsymbol\theta_{\lambda_1|\lambda_0},\boldsymbol\gamma_{\lambda_1|\lambda_0})$ can be derived:

\begin{theorem}
    \label{thm:1.1}
    For any $\lambda_1<\lambda_{0}\in (0,\lambda_{max})$, assuming $(\boldsymbol\theta_{\lambda_0},\boldsymbol\gamma_{\lambda_0})$ is known, $(\boldsymbol\theta_{\lambda_1|\lambda_0},\boldsymbol\gamma_{\lambda_1|\lambda_0})$ is bounded in a ball with center and radius
    \begin{gather}
        \begin{aligned}
            \boldsymbol c_1\equiv\binom{\boldsymbol c_1^\theta}{\boldsymbol c_2^\gamma}&=\binom{\frac{1}{2}c(1-\rho)\boldsymbol y+(1+\frac{1}{2}c\rho\lambda_0)\boldsymbol\theta_{\lambda_0}}{(1+\frac{1}{2}c\rho\lambda_0)\boldsymbol\gamma_{\lambda_0}},\\
            r_1&=\frac{c}{2}\sqrt{||\boldsymbol y||_2^2-\rho \boldsymbol y^T(\boldsymbol y-\lambda_0\boldsymbol\theta_{\lambda_0})},
        \end{aligned}
    \end{gather}
    where
    \begin{gather}
        \begin{aligned}
            c&\equiv\frac{\lambda_0-\lambda_1}{\lambda_0\lambda_1},\\
            \rho&\equiv\frac{\boldsymbol y^T(\boldsymbol y-\lambda_0\boldsymbol\theta_{\lambda_0})}{||\boldsymbol y-\lambda_0\boldsymbol\theta_{\lambda_0}||_2^2+\lambda_0^2||\boldsymbol\gamma_{\lambda_0}||_2^2}.\nonumber
        \end{aligned}
    \end{gather}
\end{theorem}

The theorem  directly implies that $\boldsymbol\theta_{\lambda_1|\lambda_0}$ is in the ball with center $\boldsymbol c_1^\theta$ and radius $r_1$.

Note if we restate $r_1$ and $\rho$ in terms of the primal solution, it becomes:

\begin{gather}
    \label{eq:thm1prim}
    \begin{aligned}
        r_1&=\frac{c}{2}\sqrt{(||\boldsymbol y||_2^2-\rho \boldsymbol y^TX\boldsymbol\beta_{\lambda_0})},\\
        \rho&\equiv\frac{\boldsymbol y^TX\boldsymbol\beta_{\lambda_0}}{||X\boldsymbol\beta_{\lambda_0}||_2^2+n(1-\alpha)\lambda_0||\boldsymbol\beta_{\lambda_0}||_2^2}.
    \end{aligned}
\end{gather}

Next, we use Theorem \ref{thm:1.1} and the fact that $(\boldsymbol\gamma_{\lambda_1|\lambda_0},\boldsymbol\theta_{\lambda_1|\lambda_0})$ is the solution to problem \ref{eq:dualmi} to derive a tighter bound for $||\boldsymbol\gamma_{\lambda_1|\lambda_0}||_2$.

\begin{theorem}
    \label{thm:1.2}
    To do.
\end{theorem}

Last, both $(\boldsymbol\theta_{\lambda_1|\lambda_0},\boldsymbol\gamma_{\lambda_1|\lambda_0})$ and $(\boldsymbol\theta_{\lambda_1},\boldsymbol\gamma_{\lambda_1})$ are the optimizer of constrained problems with the same objective function $g_{\lambda_1}$, but $(\boldsymbol\theta_{\lambda_1|\lambda_0},\boldsymbol\gamma_{\lambda_1|\lambda_0})$ is the optimizer in the set $\mathcal{F}_{\lambda_0}$, while $(\boldsymbol\theta_{\lambda_1},\boldsymbol\gamma_{\lambda_1})$ is the optimizer in the set $\mathcal{F}_{\lambda_1}$. Considering the second order expansion of $g_{\lambda_1}$ at $(\boldsymbol\theta_{\lambda_1|\lambda_0},\boldsymbol\gamma_{\lambda_1|\lambda_0})$, we can derive a bound for $(\boldsymbol\theta_{\lambda_1},\boldsymbol\gamma_{\lambda_1})$:

\begin{theorem}
    \label{thm:1.3}
    For any $\lambda_1<\lambda_{0}\in (0,\lambda_{max})$, assuming $(\boldsymbol\theta_{\lambda_1|\lambda_0},\boldsymbol\gamma_{\lambda_1|\lambda_0})$ is known, $(\boldsymbol\theta_{\lambda_1},\boldsymbol\gamma_{\lambda_1})$ is bounded in the set $\mathcal{A}^2(\lambda_1,\lambda_0|\boldsymbol\theta_{\lambda_1|\lambda_0},\boldsymbol\gamma_{\lambda_1|\lambda_0})$ such that $\boldsymbol\theta_{\lambda_1}$ is bounded in a ball with center and radius
    \begin{gather}
        \begin{aligned}
            c_2&=\boldsymbol\theta_{\lambda_1|\lambda_0}\\
            r_2&=\sqrt{c(\lambda_0-\lambda_1)}||\boldsymbol\gamma_{\lambda_1|\lambda_0}||_2.
        \end{aligned}
    \end{gather}
\end{theorem}

Bound for $\gamma_{\lambda_1}$ is not considered since it is not necessary for the rest of derivation.

\subsubsection{Method II}

The alternative method considers an alternative intermediate problem where the objective function remains the same as in the original problem at $\lambda_0$ but the feasible set changes to $\mathcal{F}_{\lambda_1}$.

\begin{gather}
        \label{eq:dualmialt}
        (\boldsymbol\theta_{\lambda_1|\lambda_0},\boldsymbol\gamma_{\lambda_1|\lambda_0})\equiv\underset{\boldsymbol\theta\in \mathbb{R}^{ n},\boldsymbol\gamma\in\mathbb{R}^p}{\mathrm{arg\,max}}g_{\lambda_0}(\boldsymbol\theta,\boldsymbol\gamma)\\
        \begin{aligned}s.t.\quad (\boldsymbol\theta,\boldsymbol\gamma)\in \mathcal{F}_{\lambda_1}\nonumber.
        \end{aligned}
\end{gather}

\begin{theorem}
    \label{thm:1.1.alt}
    For any $\lambda_1<\lambda_{0}\in (0,\lambda_{max})$, assuming $(\boldsymbol\theta_{\lambda_0},\boldsymbol\gamma_{\lambda_0})$ is known, $(\boldsymbol\theta_{\lambda_1|\lambda_0},\boldsymbol\gamma_{\lambda_1|\lambda_0})$ is bounded in the set $\mathcal{A}^1(\lambda_1,\lambda_0|\boldsymbol\theta_{\lambda_0},\boldsymbol\gamma_{\lambda_0})$ that is a ball with center and radius
    \begin{gather}
        \begin{aligned}
            \boldsymbol c_1&=\binom{\boldsymbol\theta_{\lambda_0}}{\sqrt{\frac{\lambda_1}{\lambda_0}}\boldsymbol\gamma_{\lambda_0}}\\
            r_1&=\sqrt{c(\lambda_0-\lambda_1)}||\boldsymbol\gamma_{\lambda_0}||_2.
        \end{aligned}
    \end{gather}
\end{theorem}

\begin{theorem}
    \label{thm:1.2.alt}
    For any $t\geq0$ and $\lambda_1<\lambda_{0}\in (0,\lambda_{max})$, assuming $(\boldsymbol\theta_{\lambda_1|\lambda_0},\boldsymbol\gamma_{\lambda_1|\lambda_0})$ is known, $(\boldsymbol\theta_{\lambda_1},\boldsymbol\gamma_{\lambda_1})$ is bounded in the set $\mathcal{A}^2(\lambda_1,\lambda_0,t|\boldsymbol\theta_{\lambda_1|\lambda_0},\boldsymbol\gamma_{\lambda_1|\lambda_0})$ that is a ball with center and radius
    \begin{gather}
        \begin{aligned}
            \boldsymbol c_2\equiv\binom{\boldsymbol c_2^\theta}{\boldsymbol c_2^\gamma}&=\binom{\frac{1}{2}(\frac{1-t}{\lambda_0}+c)\boldsymbol y+\frac{t+1}{2}\boldsymbol\theta_{\lambda_1|\lambda_0}}{\frac{t+1}{2}\boldsymbol\gamma_{\lambda_0}},\\
            r_2&=\frac{1}{2\lambda_0}\left\Vert\binom{(1-t)(\boldsymbol y-\lambda_0\boldsymbol\theta_{\lambda_1|\lambda_0})+c\lambda_0\boldsymbol y}{(1-t)\lambda_0\boldsymbol\gamma_{\lambda_1|\lambda_0}}\right\Vert_2,
        \end{aligned}
    \end{gather}
    where
    \begin{gather}
        \begin{aligned}
            c&\equiv\frac{\lambda_0-\lambda_1}{\lambda_0\lambda_1}.\\
        \end{aligned}
    \end{gather}
\end{theorem}

\subsection{Upper Bound in the Combined Bound}

\subsubsection{Method I}

To find the bound in \eqref{eq:boundbound}, we can first consider the problem:

\begin{equation}
    \Tilde{T}_j(\lambda_1,\lambda_0;\boldsymbol\theta',\boldsymbol\gamma')\equiv\underset{(\boldsymbol\theta,\boldsymbol\gamma)\in\mathcal{A}^2(\lambda_1,\lambda_0|\boldsymbol\theta',\boldsymbol\gamma')}{\mathrm{max}}|\boldsymbol x_j^T\boldsymbol\theta|,
\end{equation}

and it can be broken into two sub-problems:

\begin{equation}
    \label{eq:ttilde}
    \Tilde{T}^\xi_j(\lambda_1,\lambda_0;\boldsymbol\theta',\boldsymbol\gamma')\equiv\underset{(\boldsymbol\theta,\boldsymbol\gamma)\in\mathcal{A}^2(\lambda_1,\lambda_0|\boldsymbol\theta',\boldsymbol\gamma')}{\mathrm{max}}\xi \boldsymbol x_j^T\boldsymbol\theta,
\end{equation}

where $\xi\in\{-1,1\}$. For any $(\boldsymbol\theta',\boldsymbol\gamma')$, $(\boldsymbol\theta,\boldsymbol\gamma)\in\mathcal{A}^2(\lambda_1,\lambda_0|\boldsymbol\theta',\boldsymbol\gamma')$ means $\boldsymbol\theta$ is in a ball with center $\boldsymbol\theta'$ and radius $\sqrt{c(\lambda_0-\lambda_1)}||\boldsymbol\gamma'||_2$ and $\xi \boldsymbol x_j^T\boldsymbol\theta$ is linear in $\boldsymbol\theta$, so the maximum can be obtained easily:

\begin{equation}
    \label{eq:ttildexi}
    \Tilde{T}^\xi_j(\lambda_1,\lambda_0;\boldsymbol\theta',\boldsymbol\gamma')=\xi \boldsymbol x_j^T\boldsymbol\theta'+||\boldsymbol x_j||_2\sqrt{c(\lambda_0-\lambda_1)}||\boldsymbol\gamma'||_2.
\end{equation}

Next, we need to maximize $\Tilde{T}^\xi_j(\lambda_1,\lambda_0;\boldsymbol\theta',\boldsymbol\gamma')$ subject to $(\boldsymbol\theta',\boldsymbol\gamma')\in\mathcal{A}^1$.

\begin{theorem}
    \label{thm:2.1}
    For any $\lambda_1<\lambda_{0}\in (0,\lambda_{max})$, $j=1,2,...,p$ and $\xi=-1,1$, assuming $(\boldsymbol\theta_{\lambda_0},\boldsymbol\gamma_{\lambda_0})$ is known,
    \begin{gather}
        \begin{aligned}
            T^\xi_j&\equiv\underset{(\boldsymbol\theta',\boldsymbol\gamma')\in\mathcal{A}^1(\lambda_1,\lambda_0|\boldsymbol\theta_{\lambda_0},\boldsymbol\gamma_{\lambda_0})}{\mathrm{max}}\Tilde{T}^\xi_j(\lambda_1,\lambda_0;\boldsymbol\theta',\boldsymbol\gamma')\\
            &=\xi \boldsymbol x_j^T \boldsymbol c_1^\theta+||\boldsymbol x_j||_2\left(\sqrt{c(\lambda_0-\lambda_1)}||\boldsymbol c_1^\gamma||_2+\sqrt{1+c(\lambda_0-\lambda_1)}r_1\right).
        \end{aligned}
    \end{gather}
\end{theorem}

\subsubsection{Method II}

The sub-problem \eqref{eq:ttilde} is still maximizing a linear function in a ball and the maximum is easy to obtain:

\begin{gather}
    \label{eq:ttildexi.alt}
    \begin{aligned}
        \Tilde{T}^\xi_j(\lambda_1,\lambda_0,t;\boldsymbol\theta',\boldsymbol\gamma')=\xi\left( \frac{1}{2}(\frac{1-t}{\lambda_0}+c)\boldsymbol x_j^T\boldsymbol y+\frac{t+1}{2}\boldsymbol x_j^T\boldsymbol\theta'\right)+\frac{||\boldsymbol x_j||_2|1-t|}{2}\left\Vert\binom{\boldsymbol\theta'-\left(\frac{1}{\lambda_0}+\frac{c}{1-t}\right)\boldsymbol y}{\boldsymbol\gamma'}\right\Vert_2,\\
    \end{aligned}
\end{gather}

where we define $0\cdot||\boldsymbol v_1+\frac{\boldsymbol v_2}{0}||_2\equiv ||\boldsymbol v_2||_2$ for any vector $\boldsymbol v_1,\boldsymbol v_2$. There is an extra parameter $t$, and this bound will be valid for all $t\geq 0$.

\begin{theorem}
    \label{thm:2.2}
    For any $\lambda_1<\lambda_{0}\in (0,\lambda_{max})$, $j=1,2,...,p$ and $\xi=-1,1$, assuming $(\boldsymbol\theta_{\lambda_0},\boldsymbol\gamma_{\lambda_0})$ is known,
    \begin{gather}
        \begin{aligned}
            T^\xi_j(\lambda_1,\lambda_0,t;\boldsymbol\theta_{\lambda_0},\boldsymbol\gamma_{\lambda_0})&\equiv\underset{(\boldsymbol\theta',\boldsymbol\gamma')\in\mathcal{A}^1(\lambda_1,\lambda_0|\boldsymbol\theta_{\lambda_0},\boldsymbol\gamma_{\lambda_0})}{\mathrm{max}}\Tilde{T}^\xi_j(\lambda_1,\lambda_0,t;\boldsymbol\theta',\boldsymbol\gamma')\\
            %&=\xi \boldsymbol x_j^T \boldsymbol c_1^\theta+||\boldsymbol x_j||_2\left(\sqrt{c(\lambda_0-\lambda_1)}||\boldsymbol c_1^\gamma||_2+\sqrt{1+c(\lambda_0-\lambda_1)}r_1\right).
        \end{aligned}
    \end{gather}
\end{theorem}

\begin{theorem}
    \label{thm:2.3}
    For any $\lambda_1<\lambda_{0}\in (0,\lambda_{max})$, $j=1,2,...,p$ and $\xi=-1,1$, assuming $(\boldsymbol\theta_{\lambda_0},\boldsymbol\gamma_{\lambda_0})$ is known, if we define
    \begin{gather}
        \begin{aligned}
            T^\xi_j(\lambda_1,\lambda_0,t;\boldsymbol\theta_{\lambda_0},\boldsymbol\gamma_{\lambda_0})\equiv\frac{\frac{1-t}{\lambda_0}+c}{2}\xi\boldsymbol x_j^T \boldsymbol y+\frac{t+1}{2}\left(\xi \boldsymbol x_j^T \boldsymbol \theta_{\lambda_0}+||\boldsymbol x_j||_2\sqrt{c(\lambda_0-\lambda_1)}||\boldsymbol\gamma_{\lambda_0}||_2\right)\\+\frac{||\boldsymbol x_j||_2}{2}\left\Vert\binom{(1-t)\theta_{\lambda_0}-\left(\frac{1-t}{\lambda_0}+c\right)\boldsymbol y}{(1-t)\sqrt{\frac{\lambda_1}{\lambda_0}}\boldsymbol\gamma_{\lambda_0}}\right\Vert_2\\
            %&=\xi \boldsymbol x_j^T \boldsymbol c_1^\theta+||\boldsymbol x_j||_2\left(\sqrt{c(\lambda_0-\lambda_1)}||\boldsymbol c_1^\gamma||_2+\sqrt{1+c(\lambda_0-\lambda_1)}r_1\right).
        \end{aligned}
    \end{gather}
    then $T^\xi_j(\lambda_1,\lambda_0,t;\boldsymbol\theta_{\lambda_0},\boldsymbol\gamma_{\lambda_0})\geq\underset{(\boldsymbol\theta',\boldsymbol\gamma')\in\mathcal{A}^1(\lambda_1,\lambda_0|\boldsymbol\theta_{\lambda_0},\boldsymbol\gamma_{\lambda_0})}{\mathrm{max}}\Tilde{T}^\xi_j(\lambda_1,\lambda_0,t;\boldsymbol\theta',\boldsymbol\gamma')$ for all $t\geq0$.
\end{theorem}

If we define

\begin{gather}
    \begin{aligned}
        \boldsymbol v_1\equiv\binom{\frac{\boldsymbol y}{\lambda_0}-\boldsymbol\theta_{\lambda_0}}{-\sqrt{\frac{\lambda_1}{\lambda_0}}\boldsymbol\gamma_{\lambda_0}}\\
        \boldsymbol v_2\equiv\binom{\frac{\boldsymbol y}{\lambda_0}-\boldsymbol\theta_{\lambda_0}+c\boldsymbol y}{-\sqrt{\frac{\lambda_1}{\lambda_0}}\boldsymbol\gamma_{\lambda_0}}\\
        \tilde{\boldsymbol x}_j\equiv\binom{\xi\boldsymbol x_j}{\boldsymbol 0_p}
    \end{aligned}
\end{gather}

then $T^\xi_j(\lambda_1,\lambda_0,t;\boldsymbol\theta_{\lambda_0},\boldsymbol\gamma_{\lambda_0})$ becomes

\begin{equation}
    \label{eq:txi.alt}
    \xi\boldsymbol x_j^T\theta_{\lambda_0}+\frac{1}{2} \tilde{\boldsymbol x}_j^T(\boldsymbol v_2-t \boldsymbol v_1)+\frac{t+1}{2}||\tilde{\boldsymbol x}_j||_2\sqrt{c(\lambda_0-\lambda_1)}||\boldsymbol\gamma_{\lambda_0}||_2+\frac{1}{2}||\tilde{\boldsymbol x}_j||_2||\boldsymbol v_2-t\boldsymbol v_1||_2.
\end{equation}

\begin{theorem}
    \label{thm:2.4}
    For any $\lambda_1<\lambda_{0}\in (0,\lambda_{max})$, $j=1,2,...,p$ and $\xi=-1,1$, assuming $(\boldsymbol\theta_{\lambda_0},\boldsymbol\gamma_{\lambda_0})$ is known, if $\boldsymbol y\neq \boldsymbol 0$,
    \begin{gather}
        \begin{aligned}
            T^\xi_j(\lambda_1,\lambda_0;\boldsymbol\theta_{\lambda_0},\boldsymbol\gamma_{\lambda_0})\equiv\underset{t\geq 0}{\mathrm{min}}\,T^\xi_j(\lambda_1,\lambda_0,t;\boldsymbol\theta_{\lambda_0},\boldsymbol\gamma_{\lambda_0})=T^\xi_j(\lambda_1,\lambda_0,t*;\boldsymbol\theta_{\lambda_0},\boldsymbol\gamma_{\lambda_0})\\
            where\quad t^*=\begin{cases}
            0,\hfill \left(\tilde{\boldsymbol x}_j^T\boldsymbol v_1-||\tilde{\boldsymbol x}_j||_2\sqrt{c(\lambda_0-\lambda_1)}||\boldsymbol\gamma_{\lambda_0}||_2\right)^2\geq ||\tilde{\boldsymbol x}_j||^2_2||\boldsymbol v_1||^2_2\\
            \left(\frac{\boldsymbol v_1^T\boldsymbol v_2}{||\boldsymbol v_1||_2^2}+\frac{\sqrt{||\boldsymbol v_1||_2^2||\boldsymbol v_2||_2^2-(\boldsymbol v_1^T\boldsymbol v_2)^2}\left(\tilde{\boldsymbol x}_j^T\boldsymbol v_1-||\tilde{\boldsymbol x}_j||_2\sqrt{c(\lambda_0-\lambda_1)}||\boldsymbol\gamma_{\lambda_0}||_2\right)}{||\boldsymbol v_1||_2^2\sqrt{||\boldsymbol x_j||_2^2||\boldsymbol v_1||_2^2-\left(\tilde{\boldsymbol x}_j^T\boldsymbol v_1-||\tilde{\boldsymbol x}_j||_2\sqrt{c(\lambda_0-\lambda_1)}||\boldsymbol\gamma_{\lambda_0}||_2\right)^2}}\right)\vee 0,\hfill\quad o.w.
            \end{cases}
        \end{aligned}
    \end{gather}
\end{theorem}

If we define $\boldsymbol r_{\lambda_0}\equiv \boldsymbol y-X\boldsymbol\beta_{\lambda_0}$ and $\hat{\boldsymbol y}_{\lambda_0}\equiv X\boldsymbol\beta_{\lambda_0}$, then the results above can be expressed in primal variables:

\begin{gather}
    \begin{aligned}
        T^\xi_j(\lambda_1,\lambda_0,t;\boldsymbol\theta_{\lambda_0},\boldsymbol\gamma_{\lambda_0})= \frac{\xi \boldsymbol x_j^T \boldsymbol r_{\lambda_0}}{\lambda_0}+ \left(\frac{1-t}{2\lambda_0}+\frac{c}{2}\right)\xi\boldsymbol x_j^T \boldsymbol y+\frac{t-1}{2\lambda_0}\xi \boldsymbol x_j^T \boldsymbol r_{\lambda_0}+\frac{t+1}{2}||\boldsymbol x_j||_2\sqrt{n(1-\alpha) c^2\lambda_1}||\boldsymbol\beta_{\lambda_0}||_2\\
        +\frac{||\boldsymbol x_j||_2}{2\lambda_0}\sqrt{(1-t)^2||\hat{\boldsymbol y}_{\lambda_0}||_2^2+c^2\lambda_0^2||\boldsymbol y||_2^2-2(1-t)c\lambda_0 \boldsymbol y^T\hat{\boldsymbol y}_{\lambda_0}+(1-t)^2n(1-\alpha)\lambda_1||\boldsymbol\beta||_2^2}
    \end{aligned}
\end{gather}

$t^*=0$ if

\begin{equation}
    \frac{1}{\lambda_0^2}\left(||\boldsymbol x_j||_2^2||\hat{\boldsymbol y}_{\lambda_0}||_2^2-(\boldsymbol x_j^T\hat{\boldsymbol y}_{\lambda_0})^2\right)+\frac{2\lambda_1-\lambda_0}{\lambda_0\lambda_1}n(1-\alpha)||\boldsymbol x_j||_2^2||\boldsymbol\beta_{\lambda_0}||_2^2+2\sqrt{c^2\lambda_1n(1-\alpha)}\boldsymbol \xi x_j^T\hat{\boldsymbol y}_{\lambda_0}||\boldsymbol x_j||_2||\boldsymbol\beta_{\lambda_0}||_2\leq 0,
\end{equation}

else $t^*=$

\begin{equation}
    \,
\end{equation}

\subsection{Sequential Safe Screening Rule}


\appendix
\appendixpage


\section{Derivation of the Dual Problem}


Introducing 2 new variables $\boldsymbol r\equiv \boldsymbol y-X\boldsymbol\beta$ and $\boldsymbol b\equiv n(1-\alpha)\lambda \boldsymbol\beta$, then the problem \eqref{eq:enet} becomes:

\begin{equation}
    \label{eq:dual+rb}
    \begin{gathered}
    \underset{\boldsymbol\beta\in \mathbb{R}^p}{\mathrm{min}}\frac{1}{2}||\boldsymbol r||_2^2+\frac{1}{2n(1-\alpha)\lambda}||\boldsymbol b||_2^2+n\alpha\lambda||\boldsymbol\beta||_1\\s.t.\quad \boldsymbol r=\boldsymbol y-X\boldsymbol\beta,\quad \boldsymbol b=n(1-\alpha)\lambda \boldsymbol\beta.
\end{gathered}
\end{equation}

Introducing the dual variables $\boldsymbol u\in\mathbb{R}^{n},\boldsymbol w\in\mathbb{R}^p$, the dual problem becomes:

\begin{gather}
    \label{eq:dual+uw}
    \begin{aligned}
        &\underset{\boldsymbol u,\boldsymbol w}{\mathrm{max}}\,\underset{\boldsymbol r,\boldsymbol b}{\mathrm{min}}\,\underset{\boldsymbol\beta}{\mathrm{min}}\,\frac{1}{2}||\boldsymbol r||_2^2+\frac{1}{2n(1-\alpha)\lambda}||\boldsymbol b||_2^2+n\alpha\lambda||\boldsymbol\beta||_1+\boldsymbol u^T(\boldsymbol y-X\boldsymbol\beta-\boldsymbol r)+\boldsymbol w^T\left(\boldsymbol\beta-\frac{\boldsymbol b}{n(1-\alpha)\lambda}\right)\\
        =&\underset{\boldsymbol u,\boldsymbol w}{\mathrm{max}}\,\underset{\boldsymbol r,\boldsymbol b}{\mathrm{min}}\,\underset{\boldsymbol\beta}{\mathrm{min}}\,n\alpha\lambda||\boldsymbol\beta||_1-\boldsymbol u^TX\boldsymbol\beta+\boldsymbol w^T\boldsymbol\beta+\frac{1}{2}||\boldsymbol r||_2^2+\boldsymbol u^T(\boldsymbol y-\boldsymbol r)+\frac{1}{2n(1-\alpha)\lambda}||\boldsymbol b||_2^2-\frac{\boldsymbol w^T\boldsymbol b}{n(1-\alpha)\lambda}
    \end{aligned}    
\end{gather}


Minimizing with respect to $\boldsymbol\beta$, the partial derivative is:

\begin{equation}
    \label{eq:partialbeta}
    \frac{\partial}{\partial\boldsymbol\beta}(\cdot) =-X^T\boldsymbol u+\boldsymbol w+n\alpha\lambda\frac{\partial||\boldsymbol\beta||_1}{\partial\boldsymbol\beta},
\end{equation}

so the minimum is obtained iff $||X^T\boldsymbol u-\boldsymbol w||_\infty\leq n\alpha\lambda,$ and the problem becomes:

\begin{gather}
    \label{eq:dualuw}
    \begin{aligned}
        &\underset{\boldsymbol u,\boldsymbol w}{\mathrm{max}}\,\underset{\boldsymbol r,\boldsymbol b}{\mathrm{min}}\,\frac{1}{2}||\boldsymbol r||_2^2+\boldsymbol u^T(\boldsymbol y-\boldsymbol r)+\frac{1}{2n(1-\alpha)\lambda}||\boldsymbol b||_2^2-\frac{\boldsymbol w^T\boldsymbol b}{n(1-\alpha)\lambda}\\
        =&\underset{\boldsymbol u,\boldsymbol w}{\mathrm{max}}\,\underset{\boldsymbol r,\boldsymbol b}{\mathrm{min}}\,\frac{1}{2}||\boldsymbol r-\boldsymbol u||_2^2+\boldsymbol u^T\boldsymbol y-\frac{1}{2}||\boldsymbol u||_2^2+\frac{1}{2n(1-\alpha)\lambda}||\boldsymbol b-\boldsymbol w||_2^2-\frac{1}{2n(1-\alpha)\lambda}||\boldsymbol w||_2^2\\
        =&\underset{\boldsymbol u,\boldsymbol w}{\mathrm{max}}\,\underset{\boldsymbol r,\boldsymbol b}{\mathrm{min}}\,\frac{1}{2}||\boldsymbol r-\boldsymbol u||_2^2+\frac{1}{2}||\boldsymbol y||_2^2-\frac{1}{2}||\boldsymbol u-\boldsymbol y||_2^2+\frac{1}{2n(1-\alpha)\lambda}||\boldsymbol b-\boldsymbol w||_2^2-\frac{1}{2n(1-\alpha)\lambda}||\boldsymbol w||_2^2\\
        =&\underset{\boldsymbol u,\boldsymbol w}{\mathrm{max}}\,\frac{1}{2}||\boldsymbol y||_2^2-\frac{1}{2}||\boldsymbol u-\boldsymbol y||_2^2-\frac{1}{2n(1-\alpha)\lambda}||\boldsymbol w||_2^2,
    \end{aligned}
\end{gather}

where the minimum is obtained iff $\boldsymbol r=\boldsymbol u$ and $\boldsymbol b=\boldsymbol w$. Let $\boldsymbol\theta\equiv\frac{\boldsymbol u}{\lambda}=\frac{\boldsymbol y-X\boldsymbol\beta}{\lambda}$ and $\boldsymbol\gamma\equiv\frac{\boldsymbol w}{\sqrt{n(1-\alpha)\lambda^3}}=\sqrt{\frac{n(1-\alpha)}{\lambda}}\boldsymbol\beta$ The problem becomes the dual problem in \eqref{eq:dualtheta} and the dual solution and primal solution can be connected by \eqref{eq:dualprimal}.

\section{Proof of Theorem \ref{thm:1.1}}

\section{Proof of Theorem \ref{thm:1.2} (To be finished)}

The Lagrangian of the negative of the intermediate problem \eqref{eq:dualmi} is

\begin{gather}
    \begin{aligned}
        L(\boldsymbol\theta,\boldsymbol\gamma)\equiv&-\frac{1}{2}||\boldsymbol y||_2^2+\frac{\lambda_1^2}{2}\left\Vert\boldsymbol\theta-\frac{\boldsymbol y}{\lambda_1}\right\Vert_2^2+\frac{\lambda_1^2}{2}||\boldsymbol\gamma||_2^2\\
        &+\boldsymbol\eta^{+T}(X^T\boldsymbol\theta-\sqrt{n(1-\alpha)\lambda_0}\boldsymbol\gamma-n\alpha\mathbf{1}_p)+\boldsymbol\eta^{-T}(-X^T\boldsymbol\theta+\sqrt{n(1-\alpha)\lambda_0}\boldsymbol\gamma-n\alpha\mathbf{1}_p)
    \end{aligned}
\end{gather}

where $\boldsymbol\eta_+,\boldsymbol\eta_-\in \mathcal{R}^p_+$ are vectors of non-negative Lagrangian multipliers.

For any $j$, the KKT conditions take the derivative with respect to $\boldsymbol\gamma_j$ and set to 0:

\begin{equation}
    \label{eq:1.2.1}
    \frac{\partial L}{\partial \boldsymbol\gamma_j}=\lambda_1^2\boldsymbol\gamma_j-\sqrt{n(1-\alpha)\lambda_0}(\boldsymbol\eta_j^+-\boldsymbol\eta_j^-)=0.
\end{equation}

By complementary slackness, $\boldsymbol\eta_j^+>0\implies \boldsymbol x_j^T\boldsymbol\theta-\sqrt{n(1-\alpha)\lambda_0}\boldsymbol\gamma_j=n\alpha$ and $\boldsymbol\eta_j^->0\implies \boldsymbol x_j^T\boldsymbol\theta-\sqrt{n(1-\alpha)\lambda_0}\boldsymbol\gamma_j=-n\alpha$, so $\boldsymbol\eta_j^+>0$ and $\boldsymbol\eta_j^->0$ in \eqref{eq:1.2.1} cannot be non-zero at the same time, which implies:

\begin{gather}
    \begin{aligned}
        \boldsymbol\gamma_j>0\iff\boldsymbol\eta_j^+>0\implies \boldsymbol\gamma_j=\frac{\boldsymbol x_j^T\boldsymbol\theta-n\alpha}{\sqrt{n(1-\alpha)\lambda_0}},\\
        \boldsymbol\gamma_j<0\iff\boldsymbol\eta_j^->0\implies \boldsymbol\gamma_j=\frac{\boldsymbol x_j^T\boldsymbol\theta+n\alpha}{\sqrt{n(1-\alpha)\lambda_0}}.
    \end{aligned}
\end{gather}

This implies

\begin{equation}
    \label{eq:1.2.2}
    |\boldsymbol\gamma_j|\leq \max\{\boldsymbol x_j^T\boldsymbol\theta-n\alpha,-\boldsymbol x_j^T\boldsymbol\theta-n\alpha,0\}.
\end{equation}

The Slater's condition for the negative of the intermediate problem holds, so the solution $(\boldsymbol\theta_{\lambda_1|\lambda_0},\boldsymbol\gamma_{\lambda_1|\lambda_0})$ will satisfy the result \eqref{eq:1.2.2} derived from the KKT conditions. Next we consider bounding the term $\boldsymbol x_j^T\boldsymbol\theta_{\lambda_1|\lambda_0}$.

\section{Proof of Theorem \ref{thm:1.3}}

Considering the second order expansion of $g_{\lambda_1}$ at $(\boldsymbol\theta_{\lambda_1|\lambda_0},\boldsymbol\gamma_{\lambda_1|\lambda_0})$, $g_{\lambda_1}\left(\boldsymbol\theta_{\lambda_1},\sqrt{\frac{\lambda_1}{\lambda_0}}\boldsymbol\gamma_{\lambda_1}\right)$ can be written as

\begin{equation}
    \label{eq:1.3.1}
    g_{\lambda_1}\binom{\boldsymbol\theta_{\lambda_1}}{\sqrt{\frac{\lambda_1}{\lambda_0}}\boldsymbol\gamma_{\lambda_1}}=g_{\lambda_1}\binom{\boldsymbol\theta_{\lambda_1|\lambda_0}}{\boldsymbol\gamma_{\lambda_1|\lambda_0}}+\left\langle\nabla g_{\lambda_1}\binom{\boldsymbol\theta_{\lambda_1|\lambda_0}}{\boldsymbol\gamma_{\lambda_1|\lambda_0}},\binom{\boldsymbol\theta_{\lambda_1}}{\sqrt{\frac{\lambda_1}{\lambda_0}}\boldsymbol\gamma_{\lambda_1}}-\binom{\boldsymbol\theta_{\lambda_1|\lambda_0}}{\boldsymbol\gamma_{\lambda_1|\lambda_0}}\right\rangle-\frac{\lambda_1^2}{2}\left\Vert\binom{\boldsymbol\theta_{\lambda_1}}{\sqrt{\frac{\lambda_1}{\lambda_0}}\boldsymbol\gamma_{\lambda_1}}-\binom{\boldsymbol\theta_{\lambda_1|\lambda_0}}{\boldsymbol\gamma_{\lambda_1|\lambda_0}}\right\Vert_2^2,
\end{equation}

because $\nabla^2g(\lambda_1)$ is $-\lambda_1^2$ times the identity matrix. 

First, both $(\boldsymbol\theta_{\lambda_1|\lambda_0},\boldsymbol\gamma_{\lambda_1|\lambda_0})$ and $\left(\boldsymbol\theta_{\lambda_1},\sqrt{\frac{\lambda_1}{\lambda_0}}\boldsymbol\gamma_{\lambda_1}\right)$ are in the convex set $\mathcal{F}_{\lambda_0}$, because

\begin{equation}
    \left\Vert X^T\boldsymbol\theta_{\lambda_1}-\sqrt{n(1-\alpha)\lambda_0}\sqrt{\frac{\lambda_1}{\lambda_0}}\boldsymbol\gamma_{\lambda_1}\right\Vert_\infty= \left\Vert X^T\boldsymbol\theta_{\lambda_1}-\sqrt{n(1-\alpha)\lambda_1}\boldsymbol\gamma_{\lambda_1}\right\Vert_\infty\leq n\alpha.
\end{equation}

$(\boldsymbol\theta_{\lambda_1|\lambda_0},\boldsymbol\gamma_{\lambda_1|\lambda_0})$ is the maximizer in the convex set. That implies that

\begin{equation}
    \label{eq:1.3.2}
    \left\langle\nabla g_{\lambda_1}\binom{\boldsymbol\theta_{\lambda_1|\lambda_0}}{\boldsymbol\gamma_{\lambda_1|\lambda_0}},\binom{\boldsymbol\theta_{\lambda_1}}{\sqrt{\frac{\lambda_1}{\lambda_0}}\boldsymbol\gamma_{\lambda_1}}-\binom{\boldsymbol\theta_{\lambda_1|\lambda_0}}{\boldsymbol\gamma_{\lambda_1|\lambda_0}}\right\rangle\leq 0.
\end{equation}

Second, by the same argument, both $(\boldsymbol\theta_{\lambda_1},\boldsymbol\gamma_{\lambda_1})$ and $\left(\boldsymbol\theta_{\lambda_1|\lambda_0},\sqrt{\frac{\lambda_0}{\lambda_1}}\boldsymbol\gamma_{\lambda_1|\lambda_0}\right)$ are in the convex set $\mathcal{F}_{\lambda_1}$. Then,

\begin{gather}
    \label{eq:1.3.3}
    \begin{aligned}
        g_{\lambda_1}\binom{\boldsymbol\theta_{\lambda_1}}{\sqrt{\frac{\lambda_1}{\lambda_0}}\boldsymbol\gamma_{\lambda_1}}&=\frac{1}{2}||\boldsymbol y||_2^2-\frac{\lambda_1^2}{2}\left\Vert\frac{\boldsymbol y}{\lambda_1}-\boldsymbol\theta_{\lambda_1}\right\Vert_2^2-\frac{\lambda_1^3}{2\lambda_0}||\boldsymbol\gamma_{\lambda_1}||_2^2\\
        &\geq \frac{1}{2}||\boldsymbol y||_2^2-\frac{\lambda_1^2}{2}\left\Vert\frac{\boldsymbol y}{\lambda_1}-\boldsymbol\theta_{\lambda_1}\right\Vert_2^2-\frac{\lambda_1^2}{2}||\boldsymbol\gamma_{\lambda_1}||_2^2+\left(\frac{\lambda_1^2}{2}-\frac{\lambda_1^3}{2\lambda_0}\right)||\boldsymbol\gamma_{\lambda_1}||_2^2\\
        &=g_{\lambda_1}\binom{\boldsymbol\theta_{\lambda_1}}{\boldsymbol\gamma_{\lambda_1}}+\frac{c\lambda_1^3}{2}||\boldsymbol\gamma_{\lambda_1}||_2^2\\
        &\geq g_{\lambda_1}\binom{\boldsymbol\theta_{\lambda_1|\lambda_0}}{\sqrt{\frac{\lambda_0}{\lambda_1}}\boldsymbol\gamma_{\lambda_1|\lambda_0}}+\frac{c\lambda_1^3}{2}||\boldsymbol\gamma_{\lambda_1}||_2^2\\
        &=g_{\lambda_1}\binom{\boldsymbol\theta_{\lambda_1|\lambda_0}}{\boldsymbol\gamma_{\lambda_1|\lambda_0}}-\frac{c\lambda_0\lambda_1^2}{2}||\boldsymbol\gamma_{\lambda_1|\lambda_0}||_2^2+\frac{c\lambda_1^3}{2}||\boldsymbol\gamma_{\lambda_1}||_2^2,
    \end{aligned}
\end{gather}

where the inequality is due to the fact that $(\boldsymbol\theta_{\lambda_1},\boldsymbol\gamma_{\lambda_1})$ is the maximizer in $\mathcal{F}_{\lambda_1}$. Combining \eqref{eq:1.3.1}, \eqref{eq:1.3.2} and \eqref{eq:1.3.3} we have:

\begin{gather}
    \begin{aligned}
        &\left\Vert\binom{\boldsymbol\theta_{\lambda_1}}{\sqrt{\frac{\lambda_1}{\lambda_0}}\boldsymbol\gamma_{\lambda_1}}-\binom{\boldsymbol\theta_{\lambda_1|\lambda_0}}{\boldsymbol\gamma_{\lambda_1|\lambda_0}}\right\Vert_2^2\leq c\lambda_0||\boldsymbol\gamma_{\lambda_1|\lambda_0}||_2^2-c\lambda_1||\boldsymbol\gamma_{\lambda_1}||_2^2,
    \end{aligned}
\end{gather}

and rearranging the terms yields the result in Theorem \ref{thm:1.3}:

\begin{gather}
    \begin{aligned}
        ||\boldsymbol\theta_{\lambda_1}-\boldsymbol\theta_{\lambda_1|\lambda_0}||_2^2&\leq c\lambda_0||\boldsymbol\gamma_{\lambda_1|\lambda_0}||_2^2-c\lambda_1||\boldsymbol\gamma_{\lambda_1}||_2^2-\left\Vert\sqrt{\frac{\lambda_1}{\lambda_0}}\boldsymbol\gamma_{\lambda_1}-\boldsymbol\gamma_{\lambda_1|\lambda_0}\right\Vert_2^2\\
        &=c\lambda_0||\boldsymbol\gamma_{\lambda_1|\lambda_0}||_2^2-c\lambda_1||\boldsymbol\gamma_{\lambda_1}||_2^2-\frac{\lambda_1}{\lambda_0}||\boldsymbol\gamma_{\lambda_1}||_2^2+2\sqrt{\frac{\lambda_1}{\lambda_0}}\boldsymbol\gamma_{\lambda_1}^T\boldsymbol\gamma_{\lambda_1|\lambda_0}-||\boldsymbol\gamma_{\lambda_1|\lambda_0}||_2^2\\
        &=-||\boldsymbol\gamma_{\lambda_1}||_2^2+2\sqrt{\frac{\lambda_1}{\lambda_0}}\boldsymbol\gamma_{\lambda_1}^T\gamma_{\lambda_1|\lambda_0}+(c\lambda_0-1)||\boldsymbol\gamma_{\lambda_1|\lambda_0}||_2^2\\
        &=-\left\Vert\boldsymbol\gamma_{\lambda_1}-\sqrt{\frac{\lambda_1}{\lambda_0}}\boldsymbol\gamma_{\lambda_1|\lambda_0}\right\Vert_2^2+c(\lambda_0-\lambda_1)||\boldsymbol\gamma_{\lambda_1|\lambda_0}||_2^2\\
        &\leq c(\lambda_0-\lambda_1)||\boldsymbol\gamma_{\lambda_1|\lambda_0}||_2^2.
    \end{aligned}
\end{gather}

\section{Proof of Theorem \ref{thm:2.1}}

$T_j^\xi$ is the maximum of the problem

\begin{gather}
    \begin{aligned}
        &\underset{\theta',\gamma'}{\mathrm{max}}\,\xi \boldsymbol x_j^T\boldsymbol\theta'+||\boldsymbol x_j||_2\sqrt{c(\lambda_0-\lambda_1)}||\boldsymbol\gamma'||_2\\
        &s.t.\quad \left\Vert\binom{\boldsymbol\theta'}{\boldsymbol\gamma'}-\binom{\boldsymbol c_1^\theta}{\boldsymbol c_1^\gamma}\right\Vert^2_2\leq r_1^2.
    \end{aligned}
\end{gather}

Let $r^\theta=||\boldsymbol\theta'-\boldsymbol c_1^\theta||_2$ and $r^\gamma=||\boldsymbol\gamma'-\boldsymbol c_1^\gamma||_2$. The problem becomes

\begin{gather}
    \begin{aligned}
        &\underset{\boldsymbol\theta',\boldsymbol\gamma',r^\theta,r^\gamma}{\mathrm{max}}\,\xi \boldsymbol x_j^T\boldsymbol\theta'+||\boldsymbol x_j||_2\sqrt{c(\lambda_0-\lambda_1)}||\boldsymbol\gamma'||_2\\
        &s.t.\quad r^\theta=||\boldsymbol\theta'-\boldsymbol c_1^\theta||_2,\,r^\gamma=||\boldsymbol\gamma'-\boldsymbol c_1^\gamma||_2,\,r^{\theta2}+r^{\gamma2}=r_1^2.
    \end{aligned}
\end{gather}

Maximizing with respect to $\boldsymbol\theta'$ and $\boldsymbol\gamma'$ can be easily done since they can be maximized independently and their constraints are both ball constraints. The problems becomes

\begin{gather}
    \begin{aligned}
        &\underset{r^\theta,r^\gamma}{\mathrm{max}}\,\xi \boldsymbol x_j^T\boldsymbol c_1^\theta+||\boldsymbol x_j||_2\left(r^\theta+\sqrt{c(\lambda_0-\lambda_1)}(||\boldsymbol c_1^\gamma||_2+r^\gamma)\right)\\
        &s.t.\quad r^{\theta2}+r^{\gamma2}=r_1^2.
    \end{aligned}
\end{gather}

Again, it is a maximization in a ball constraint of $(r^\theta,r^\gamma)$ and the maximum will be the form in \ref{thm:1.3}. 

\section{Proof of Theorem \ref{thm:2.2}}

Defining the following quantities:
\begin{gather}
    \begin{aligned}
        \boldsymbol b_0&\equiv\binom{\frac{t+1}{2}\xi \boldsymbol x_j}{\boldsymbol 0},\\
        a_0&\equiv\frac{||\boldsymbol x_j||_2|1-t|)}{2},\\
        \boldsymbol c_0'&\equiv\binom{\left(\frac{1}{\lambda_0}+\frac{c}{1-t}\right)\boldsymbol y}{0},\\
        \boldsymbol z &\equiv \binom{\boldsymbol\theta'}{\boldsymbol\gamma'}.
    \end{aligned}
\end{gather}

Note $\boldsymbol c_1-\boldsymbol c_0'=\binom{*}{\sqrt{\frac{\lambda_1}{\lambda_0}}\boldsymbol\gamma_{\lambda_0}}$, where the last $p$ elements will be non-zero when $\beta_{\lambda_0}$ is non-zero, which will be true because $\lambda_0<\lambda_{\max}$. This shows that $\boldsymbol c_1-\boldsymbol c_0'$ and $\boldsymbol b_0$ will never be colinear.

The problem \eqref{eq:ttildexi.alt} is equivalent to maximizing
\begin{equation}
    \label{eq:2.2.1}
    \boldsymbol b_0^T\boldsymbol z+a_0||\boldsymbol z-\boldsymbol c_0'||_2
\end{equation}

subject to the ball constraint $||\boldsymbol z-\boldsymbol c_1||_2^2\leq r_1^2$.
Take derivative with respect to $\boldsymbol z$:
\begin{equation}
    \label{eq:2.2.2}
    \frac{\partial}{\partial\boldsymbol z}=\boldsymbol b_0^T+a_0\frac{\boldsymbol z-\boldsymbol c_0'}{||\boldsymbol z-\boldsymbol c_0'||_2}.
\end{equation}
\begin{enumerate}
    \item If $t>0$:
    
    The norm of the derivative \eqref{eq:2.2.2} is positive
    \begin{equation}
        \left\Vert\frac{\partial}{\partial\boldsymbol z}=\boldsymbol b_0^T+a_0\frac{\boldsymbol z-\boldsymbol c_0'}{||\boldsymbol z-\boldsymbol c_0'||_2}\right\Vert_2^2\geq ||\boldsymbol b_0||_2^2-|a_0|\left\Vert\frac{\boldsymbol z-\boldsymbol c_0'}{||\boldsymbol z-\boldsymbol c_0'||_2}\right\Vert_2^2=\frac{t+1-|1-t|}{2}||\boldsymbol x_j||_2^2>0,
    \end{equation}
    which means the maximum will not be obtained in the interior of the ball and can only be obtained on the boundary. The Lagrangian of \eqref{eq:2.2.1} is
    \begin{equation}
        L=\boldsymbol b_0^Tz+a_0||\boldsymbol z-\boldsymbol c'_0||_2-s(||\boldsymbol z-\boldsymbol c_1||_2-r_1),
    \end{equation}
    where $s\geq0$ is the Lagrangian multiplier. Take derivative with respect to $\boldsymbol z$ and set to 0:
    \begin{gather}
        \begin{aligned}
            &\frac{\partial L}{\partial \boldsymbol z}=\boldsymbol b_0+a_0\frac{\boldsymbol z-\boldsymbol c_0'}{||\boldsymbol z-\boldsymbol c_0'||_2}-s\frac{\boldsymbol z-\boldsymbol c_1}{||\boldsymbol z-\boldsymbol c_1||_2}=0\\
            \implies &\left(\frac{s}{||\boldsymbol z-\boldsymbol c_1||_2}-\frac{a_0}{||\boldsymbol z-\boldsymbol c_0'||_2}\right)(\boldsymbol z-\boldsymbol c_1)=\boldsymbol b_0+\frac{a_0(\boldsymbol c_1-\boldsymbol c_0')}{||\boldsymbol z- \boldsymbol c_0'||_2}.
        \end{aligned}
    \end{gather}
    
    Because $\boldsymbol b_0$ and $(\boldsymbol c_1-\boldsymbol c_0')$ are not colinear, the right hand side cannot be zero and thus the left hand side cannot be zero. As a result $(\boldsymbol z- \boldsymbol c_1)$ will be in the space spanned by $\boldsymbol b_0$ and $(\boldsymbol c_1-\boldsymbol c_0')$. Combine with the fact that $\boldsymbol z$ will be on the boundary of the ball, we can decompose $\boldsymbol z$ as two orthogonal parts 
    \begin{gather}
        \begin{aligned}
            &\boldsymbol z=\boldsymbol c_1+w_0 r_1\boldsymbol v_0+w_1 r_1\boldsymbol v_1,\\
            where&\quad \boldsymbol v_0\equiv\frac{\boldsymbol b_0}{||\boldsymbol b_0||_2},\, \boldsymbol v_1\equiv \frac{(\boldsymbol c_1-\boldsymbol c_0')-\frac{(\boldsymbol c_1-\boldsymbol c_0')^T\boldsymbol b_0}{||\boldsymbol b_0||_2^2}\boldsymbol b_0}{||(\boldsymbol c_1-\boldsymbol c_0')-\frac{(\boldsymbol c_1-\boldsymbol c_0')^T\boldsymbol b_0}{||\boldsymbol b_0||_2^2}\boldsymbol b_0||_2}\\
            s.t.&\quad w_0^2+w_1^2=1.
        \end{aligned}
    \end{gather}
\end{enumerate}

\section{Proof of Theorem \ref{thm:2.4}}

\begin{lemma}
    \label{lem:2.4.1}
    $\boldsymbol v_1^T \boldsymbol v_2\geq 0$.
\end{lemma}

It is also clear that $\boldsymbol v_1$ and $\boldsymbol v_2$ are not colinear as long as $\boldsymbol y\neq \boldsymbol0$. Take derivative of \eqref{eq:txi.alt} with respect to $t$ and set to 0.

\begin{gather}
    \label{eq:2.4.1}
    \begin{aligned}
        &\frac{\partial}{\partial t}=-\frac{1}{2}\tilde{\boldsymbol x}_j^T\boldsymbol v_1+\frac{1}{2}||\tilde{\boldsymbol x}_j||_2\sqrt{c(\lambda_0-\lambda_1)}||\boldsymbol\gamma_{\lambda_0}||_2-\frac{1}{2}||\tilde{\boldsymbol x}_j||_2\frac{(\boldsymbol v_2-t\boldsymbol v_1)^T\boldsymbol v_1}{||\boldsymbol v_2-t\boldsymbol v_1||_2}=0\\
        \implies & \left(-\tilde{\boldsymbol x}_j^T\boldsymbol v_1+||\tilde{\boldsymbol x}_j||_2\sqrt{c(\lambda_0-\lambda_1)}||\boldsymbol\gamma_{\lambda_0}||_2\right)||\boldsymbol v_2-t\boldsymbol v_1||_2=||\tilde{\boldsymbol x}_j||_2(\boldsymbol v_2-t\boldsymbol v_1)^T\boldsymbol v_1
    \end{aligned}
\end{gather}

If $-\tilde{\boldsymbol x}_j^T\boldsymbol v_1+||\tilde{\boldsymbol x}_j||_2\sqrt{c(\lambda_0-\lambda_1)}||\boldsymbol\gamma_{\lambda_0}||_2\geq ||\tilde{\boldsymbol x}_j||_2||\boldsymbol v_1||_2$, the minimum will be obtained at $t=0$. Else, we have $-\tilde{\boldsymbol x}_j^T\boldsymbol v_1+||\tilde{\boldsymbol x}_j||_2\sqrt{c(\lambda_0-\lambda_1)}||\boldsymbol\gamma_{\lambda_0}||_2>- ||\tilde{\boldsymbol x}_j||_2||\boldsymbol v_1||_2$ by Cauchy-Schwartz, and squaring both sides of \eqref{eq:2.4.1} and simplify it:

\begin{gather}
    \begin{aligned}
        \left(\left(-\tilde{\boldsymbol x}_j^T\boldsymbol v_1+||\tilde{\boldsymbol x}_j||_2\sqrt{c(\lambda_0-\lambda_1)}||\boldsymbol\gamma_{\lambda_0}||_2\right)^2-||\boldsymbol x_j||_2^2||\boldsymbol v_1||_2^2\right)\left(||\boldsymbol v_1||_2^2t^2-2\boldsymbol v_1^T\boldsymbol v_2 t+||\boldsymbol v_2||_2^2\right)\\
        +||\boldsymbol x_j||_2^2(||\boldsymbol v_1||_2^2||\boldsymbol v_2||_2^2-(\boldsymbol v_1^T\boldsymbol v_2)^2)=0.
    \end{aligned}
\end{gather}

\eqref{eq:2.4.1} also implies that

\begin{gather}
    \begin{aligned}
        &\left(\tilde{\boldsymbol x}_j^T\boldsymbol v_1-||\tilde{\boldsymbol x}_j||_2\sqrt{c(\lambda_0-\lambda_1)}||\boldsymbol\gamma_{\lambda_0}||_2\right)\boldsymbol v_1(t\boldsymbol v_1-\boldsymbol v_2)^T\boldsymbol v_1\geq 0\\
        \implies&\begin{cases}
        t\geq\frac{\boldsymbol v_1^T\boldsymbol v_2}{||\boldsymbol v_1||_2^2},\quad \textit{if}\quad\tilde{\boldsymbol x}_j^T\boldsymbol v_1-||\tilde{\boldsymbol x}_j||_2\sqrt{c(\lambda_0-\lambda_1)}||\boldsymbol\gamma_{\lambda_0}||_2 >0\\
        t\leq\frac{\boldsymbol v_1^T\boldsymbol v_2}{||\boldsymbol v_1||_2^2},\quad \textit{if}\quad\tilde{\boldsymbol x}_j^T\boldsymbol v_1-||\tilde{\boldsymbol x}_j||_2\sqrt{c(\lambda_0-\lambda_1)}||\boldsymbol\gamma_{\lambda_0}||_2<0,
        \end{cases}
    \end{aligned}
\end{gather}

so the solution to \eqref{eq:2.4.1} will be:

\begin{equation}
    t=\left(\frac{\boldsymbol v_1^T\boldsymbol v_2}{||\boldsymbol v_1||_2^2}+\sqrt{\frac{||\boldsymbol v_1||_2^2||\boldsymbol v_2||_2^2-(\boldsymbol v_1^T\boldsymbol v_2)^2}{||\boldsymbol x_j||_2^2||\boldsymbol v_1||_2^2-\left(\tilde{\boldsymbol x}_j^T\boldsymbol v_1-||\tilde{\boldsymbol x}_j||_2\sqrt{c(\lambda_0-\lambda_1)}||\boldsymbol\gamma_{\lambda_0}||_2\right)^2}}\frac{\tilde{\boldsymbol x}_j^T\boldsymbol v_1-||\tilde{\boldsymbol x}_j||_2\sqrt{c(\lambda_0-\lambda_1)}||\boldsymbol\gamma_{\lambda_0}||_2}{||\boldsymbol v_1||_2^2}\right)\vee 0.
\end{equation}

\section{Proof of Lemme \ref{lem:2.4.1}}

Let

\begin{gather}
    \begin{aligned}
        \boldsymbol v_1^*\equiv\binom{\frac{\boldsymbol y}{\lambda_0}-\boldsymbol\theta_{\lambda_0}}{-\boldsymbol\gamma_{\lambda_0}}\\
        \boldsymbol v_2^*\equiv\binom{\frac{\boldsymbol y}{\lambda_0}-\boldsymbol\theta_{\lambda_0}+c\boldsymbol y}{-\boldsymbol\gamma_{\lambda_0}}\\
    \end{aligned}
\end{gather}

Because for all $t\geq 0$, $(\boldsymbol\theta_{\lambda_0},\boldsymbol\gamma_{\lambda_0})$ is the projection of $(\boldsymbol\theta_{\lambda_0},\boldsymbol\gamma_{\lambda_0})+t\boldsymbol v_1^*$ onto $\mathcal{F}_{\lambda_0}$ and $\boldsymbol 0\in \mathcal{F}_{\lambda_0}$, the distance between $(\boldsymbol\theta_{\lambda_0},\boldsymbol\gamma_{\lambda_0})$ and $(\boldsymbol\theta_{\lambda_0},\boldsymbol\gamma_{\lambda_0})+t\boldsymbol v_1^*$ will be no larger than the distance between $(\boldsymbol\theta_{\lambda_0},\boldsymbol\gamma_{\lambda_0})+t\boldsymbol v_1^*$ and $\boldsymbol 0$:

\begin{gather}
    \begin{aligned}
        ||t\boldsymbol v_1^*||_2^2\leq ||(\boldsymbol\theta_{\lambda_0},\boldsymbol\gamma_{\lambda_0})+t\boldsymbol v_1^*||_2^2=||t\boldsymbol v_1^*||_2^2+||(\boldsymbol\theta_{\lambda_0},\boldsymbol\gamma_{\lambda_0})||_2^2+2t(\boldsymbol\theta_{\lambda_0},\boldsymbol\gamma_{\lambda_0})^T\boldsymbol v_1^*.
    \end{aligned}
\end{gather}

It holds for all $t\geq 0$, which means

\begin{gather}
    \begin{aligned}
        &0\leq\binom{\boldsymbol\theta_{\lambda_0}}{\boldsymbol\gamma_{\lambda_0}}^T\boldsymbol v_1^*=\frac{\boldsymbol y^T\boldsymbol\theta_{\lambda_0}}{\lambda_0}-||\boldsymbol\theta_{\lambda_0}||_2^2-||\boldsymbol\gamma_{\lambda_0}||_2^2\\
        \implies&||\boldsymbol\theta_{\lambda_0}||^2_2\leq\frac{\boldsymbol y^T\boldsymbol\theta_{\lambda_0}}{\lambda_0}-||\boldsymbol\gamma_{\lambda_0}||_2^2\leq\frac{\boldsymbol y^T\boldsymbol\theta_{\lambda_0}}{\lambda_0}\leq \frac{||\boldsymbol y||_2||\boldsymbol\theta_{\lambda_0}||_2}{\lambda_0}\\
        \implies&||\boldsymbol\theta_{\lambda_0}||^2\leq\frac{||\boldsymbol y||_2}{\lambda_0}.
    \end{aligned}
\end{gather}

Last,

\begin{gather}
    \begin{aligned}
        \boldsymbol v_1^T\boldsymbol v_2&=||\frac{\boldsymbol y}{\lambda_0}-\boldsymbol\theta_{\lambda_0}||_2^2+c\boldsymbol y^T(\frac{\boldsymbol y}{\lambda_0}-\boldsymbol\theta_{\lambda_0})+\frac{\lambda_1}{\lambda_0}||\boldsymbol\gamma_{\lambda_0}||_2^2\\
        &\geq c\boldsymbol y^T(\frac{\boldsymbol y}{\lambda_0}-\boldsymbol\theta_{\lambda_0})\\
        &=c\left(\frac{||\boldsymbol y||_2^2}{\lambda_0}-\boldsymbol y^T\boldsymbol\theta_{\lambda_0}\right)\\
        &\geq c\left(\frac{||\boldsymbol y||_2^2}{\lambda_0}-||\boldsymbol y||_2||\boldsymbol\theta_{\lambda_0}||_2\right)\geq 0
    \end{aligned}
\end{gather}

\section{Enhanced EDPP}

EDPP says $\forall t\geq 0$

\begin{equation}
    \left\Vert\boldsymbol\theta_{\lambda_1}-\left(\boldsymbol\theta_{\lambda_0}+\frac{1}{2}(\boldsymbol v_2-t\boldsymbol v_1)\right)\right\Vert_2^2\leq\frac{1}{4}||\boldsymbol v_2-t\boldsymbol v_1||_2^2,
\end{equation}

where when $\lambda_0<\lambda_{\max}$, $\boldsymbol v_1=\frac{\boldsymbol y}{\lambda_0}-\boldsymbol\theta_{\lambda_0}$, $\boldsymbol v_2=\frac{\boldsymbol y}{\lambda_1}-\boldsymbol\theta_{\lambda_0}$ and $\boldsymbol v_1^T\boldsymbol v_2\geq0$. That means

\begin{gather}
    \label{eq:edppobj}
    \begin{aligned}
        \boldsymbol x_j^T\boldsymbol\theta_{\lambda_1}&\leq T^\xi_j(\lambda_1,\lambda_0,t)\equiv \boldsymbol x_j^T\boldsymbol\theta_{\lambda_0}+\frac{1}{2}\boldsymbol x_j^T(\boldsymbol v_2-t\boldsymbol v_1)+\frac{1}{2}||\boldsymbol x_j||_2||\boldsymbol v_2-t\boldsymbol v_1||_2\\
        %&=\boldsymbol x_j^T\boldsymbol\theta_{\lambda_0}+\frac{||\boldsymbol x_j||_2||\boldsymbol v_2-t\boldsymbol v_1||}{2}\left(\frac{\boldsymbol x_j^T(\boldsymbol v_2-t\boldsymbol v_1)}{||\boldsymbol x_j||_2||\boldsymbol v_2-t\boldsymbol v_1||}+1\right)
    \end{aligned}
\end{gather}

If $\boldsymbol v_1,\boldsymbol v_2$ are not colinear, which is true when $\boldsymbol y$ and $X\boldsymbol\beta_\lambda$ are not colinear, take derivative with respect to $t$ and set to 0

\begin{gather}
    \label{eq:edppdt}
    \begin{aligned}
        &\frac{\partial}{\partial t}=-\frac{1}{2}\boldsymbol x_j^T\boldsymbol v_1-\frac{1}{2}||\boldsymbol x_j||_2\frac{(\boldsymbol v_2-t\boldsymbol v_1)^T\boldsymbol v_1}{||\boldsymbol v_2-t\boldsymbol v_1||_2}=0\\
        \implies & -\frac{\boldsymbol x_j^T\boldsymbol v_1}{||\boldsymbol x_j||_2||\boldsymbol v_1||_2}=\frac{(\boldsymbol v_2-t\boldsymbol v_1)^T\boldsymbol v_1}{||\boldsymbol v_2-t\boldsymbol v_1||_2||\boldsymbol v_1||_2}\\
    \end{aligned}
\end{gather}

Take the second derivative:

\begin{equation}
    \frac{\partial^2}{\partial t^2}=||\boldsymbol x_j||_2\frac{||\boldsymbol v_1||^2_2||\boldsymbol v_2-t\boldsymbol v_1||^2_2-\left((\boldsymbol v_2-t\boldsymbol v_1)^T\boldsymbol v_1\right)^2}{2||\boldsymbol v_2-t\boldsymbol v_1||^3_2}>0
\end{equation}

If $\boldsymbol x_j$ and $\boldsymbol v_1$ are positively colinear, \eqref{eq:edppobj} will be minimized when $t\xrightarrow[]{}\infty$ and the minimum is $\boldsymbol x_j^T\boldsymbol\theta_{\lambda_0}+\frac{1}{2}\boldsymbol x_j^T \boldsymbol v_2-\frac{||\boldsymbol x_j||_2 \boldsymbol v_1^T \boldsymbol v_2}{2||\boldsymbol v_1||_2 }$. If $\boldsymbol x_j$ and $\boldsymbol v_1$ are negatively colinear, \eqref{eq:edppobj} will be minimized when $t=0$. If $\boldsymbol x_j^T \boldsymbol v_1=0$, solution to \eqref{eq:edppdt} will be $\frac{\boldsymbol v_1^T \boldsymbol v_2}{||\boldsymbol v_1||_2^2}$. Else, square the two terms \eqref{eq:edppdt} and set them to equal:

\begin{gather}
    \begin{aligned}
        \label{eq:edppquad}
        &(\boldsymbol x_j^T \boldsymbol v_1)^2||\boldsymbol v_2-t\boldsymbol v_1||_2^2=\left((\boldsymbol v_2-t\boldsymbol v_1)^T \boldsymbol v_1\right)^2||\boldsymbol x_j||_2^2\\
        \implies&\left((\boldsymbol x_j^T\boldsymbol v_1)^2-||\boldsymbol x_j||_2^2||\boldsymbol v_1||_2^2\right)(\boldsymbol v_2-t\boldsymbol v_1)^2+||\boldsymbol x_j||_2^2\left(||\boldsymbol v_1||_2^2||\boldsymbol v_2||_2^2-(\boldsymbol v_1^T\boldsymbol v_2)^2\right)=0\\
        \implies&\left((\boldsymbol x_j^T\boldsymbol v_1)^2-||\boldsymbol x_j||_2^2||\boldsymbol v_1||_2^2\right)\left(||\boldsymbol v_1||_2^2t^2-2\frac{\boldsymbol v_1^T \boldsymbol v_2}{||\boldsymbol v_1||_2}t\right)+(\boldsymbol x_j^T v_1)^2||\boldsymbol v_2||_2^2-||\boldsymbol x_j||_2^2(\boldsymbol v_1^Tv_2)^2=0.
    \end{aligned}
\end{gather}

\eqref{eq:edppdt} also implies that

\begin{gather}
    \begin{aligned}
        &\boldsymbol x_j^T\boldsymbol v_1(t\boldsymbol v_1-\boldsymbol v_2)^T\boldsymbol v_1\geq 0\\
        \implies&\begin{cases}
        t\geq\frac{\boldsymbol v_1^T\boldsymbol v_2}{||\boldsymbol v_1||_2^2},\quad \textit{if}\quad\boldsymbol x_j^T\boldsymbol v_1>0\\
        t\leq\frac{\boldsymbol v_1^T\boldsymbol v_2}{||\boldsymbol v_1||_2^2},\quad \textit{if}\quad\boldsymbol x_j^T\boldsymbol v_1<0
        \end{cases}
    \end{aligned}
\end{gather}

so the solution to \eqref{eq:edppquad} will be:

\begin{equation}
    t^*=\left(\frac{\boldsymbol v_1^T\boldsymbol v_2}{||\boldsymbol v_1||_2^2}+\sqrt{\frac{||\boldsymbol v_1||_2^2||\boldsymbol v_2||_2^2-(\boldsymbol v_1^T\boldsymbol v_2)^2}{||\boldsymbol x_j||_2^2||\boldsymbol v_1||_2^2-(\boldsymbol x_j^T\boldsymbol v_1)^2}}\frac{\boldsymbol x_j^T\boldsymbol v_1}{||\boldsymbol v_1||_2^2}\right)\vee 0.
\end{equation}

This form of solution also covers the cases when $\boldsymbol v_1,\boldsymbol v_2$ are colinear or $\boldsymbol x_j^T \boldsymbol v_1=0$.

If we define $\boldsymbol r_{\lambda_0}\equiv \boldsymbol y-X\boldsymbol\beta_{\lambda_0}$ and $\hat{\boldsymbol y}_{\lambda_0}\equiv X\boldsymbol\beta_{\lambda_0}$, then the results above can be expressed in primal variables:

\begin{gather}
    \begin{aligned}
        T^\xi_j(\lambda_1,\lambda_0,t)= \frac{\xi \boldsymbol x_j^T \boldsymbol r_{\lambda_0}}{\lambda_0}+ \frac{c}{2}\xi\boldsymbol x_j^T \boldsymbol y+\frac{1-t}{2\lambda_0}\xi \boldsymbol x_j^T \hat{\boldsymbol y}_{\lambda_0}\\
        +\frac{||\boldsymbol x_j||_2}{2\lambda_0}\sqrt{(1-t)^2||\hat{\boldsymbol y}_{\lambda_0}||_2^2+c^2\lambda_0^2||\boldsymbol y||_2^2+2(1-t)c\lambda_0 \boldsymbol y^T\hat{\boldsymbol y}_{\lambda_0}}
    \end{aligned}
\end{gather}

If $\boldsymbol y^T \hat{\boldsymbol y}_{\lambda_0}=||\boldsymbol y||_2|| \hat{\boldsymbol y}_{\lambda_0}||_2$ or $(\boldsymbol x_j^T\hat{\boldsymbol y}_{\lambda_0})^2<||\boldsymbol x_j||_2||\hat{\boldsymbol y}_{\lambda_0}||_2^2$,

\begin{equation}
    t^*=\left(1+\frac{c\lambda_0\boldsymbol y^T\hat{\boldsymbol y}_{\lambda_0}}{||\hat{\boldsymbol y}_{\lambda_0}||_2^2}+c\lambda_0\sqrt{\frac{||\boldsymbol y||_2^2||\hat{\boldsymbol y}_{\lambda_0}||_2^2-(\boldsymbol y^T\hat{\boldsymbol y}_{\lambda_0})^2}{||\boldsymbol x_j||_2^2||\hat{\boldsymbol y}_{\lambda_0}||_2^2-(\boldsymbol x_j^T\hat{\boldsymbol y}_{\lambda_0})^2}}\frac{\xi\boldsymbol x_j^T\hat{\boldsymbol y}_{\lambda_0}}{||\hat{\boldsymbol y}_{\lambda_0}||_2^2}\right)\vee 0.
\end{equation}

Else, $t^*=0$ if $\xi\boldsymbol x_j^T\hat{\boldsymbol y}_{\lambda_0}=-||\boldsymbol x_j||_2||\hat{\boldsymbol y}_{\lambda_0}||_2$ and $t^*=\infty$ if $\xi\boldsymbol x_j^T\hat{\boldsymbol y}_{\lambda_0}=||\boldsymbol x_j||_2||\hat{\boldsymbol y}_{\lambda_0}||_2$ with

\begin{equation}
    T^\xi_j(\lambda_1,\lambda_0,t)=\frac{\xi \boldsymbol x_j^T \boldsymbol r_{\lambda_0}}{\lambda_0}+\frac{c}{2}\xi\boldsymbol x_j^T\boldsymbol y-\frac{c||\boldsymbol x_j||_2\boldsymbol y^T\hat{\boldsymbol y}_{\lambda_0}}{2||\hat{\boldsymbol y}_{\lambda_0}||_2}
\end{equation}

When $\lambda_0=\lambda_{\max}=\frac{|\boldsymbol x_*^T\boldsymbol y|}{n}$, $\boldsymbol v_1=sign(\boldsymbol x_*^T\boldsymbol y)\boldsymbol x_*$ and $\boldsymbol v_2=c\boldsymbol y$. If we define $\hat{\boldsymbol y}_{\lambda_0}\equiv \lambda_0sign(\boldsymbol x_*^T\boldsymbol y) \boldsymbol x_*$

\begin{gather}
    \begin{aligned}
        T^\xi_j(\lambda_1,\lambda_0,t)= \frac{\xi \boldsymbol x_j^T \boldsymbol y}{\lambda_0}+ \frac{c}{2}\boldsymbol x_j^T\boldsymbol y-\frac{t}{2\lambda_0}\xi\boldsymbol x_j^T\hat{\boldsymbol y}_{\lambda_0}\\
        +\frac{||\boldsymbol x_j||_2}{2}\sqrt{t^2||\boldsymbol x_*||_2^2+c^2||\boldsymbol y||_2^2-2tc\lambda_0n}.
    \end{aligned}
\end{gather}

If $n\lambda_0=||\boldsymbol x_*||_2||\boldsymbol y||_2$ or $(\boldsymbol x_j^T\hat{\boldsymbol y}_{\lambda_0})^2<\lambda_0^2||\boldsymbol x_j||_2^2||\boldsymbol x_*||_2^2$

\begin{equation}
    t^*=\left(\frac{c\lambda_0n}{||\boldsymbol x_*||_2^2}+c\sqrt{\frac{||\boldsymbol y||_2^2||\boldsymbol x_*||_2^2-n^2\lambda_0^2}{\lambda_0^2||\boldsymbol x_j||_2^2||\boldsymbol x_*||_2^2-(\boldsymbol x_j^T\hat{\boldsymbol y}_{\lambda_0})^2}}\frac{\xi\boldsymbol x_j^T\hat{\boldsymbol y}_{\lambda_0}}{||\boldsymbol x_*||_2^2}\right)\vee 0.
\end{equation}

Else, $t^*=0$ if $\xi\boldsymbol x_j^T\hat{\boldsymbol y}_{\lambda_0}=-\lambda_0||\boldsymbol x_j||_2||\boldsymbol x_*||_2$ and $t^*=\infty$ if $\xi\boldsymbol x_j^T\hat{\boldsymbol y}_{\lambda_0}=\lambda_0||\boldsymbol x_j||_2||\boldsymbol x_*||_2$ with

\begin{equation}
    T^\xi_j(\lambda_1,\lambda_0,t)=\frac{\xi \boldsymbol x_j^T \boldsymbol y}{\lambda_0}+\frac{c}{2}\xi\boldsymbol x_j^T\boldsymbol y-\frac{cn\lambda_0||\boldsymbol x_j||_2}{2||\boldsymbol x_*||_2}
\end{equation}



